%-- coding: UTF-8 -- 
\documentclass[UTF8]{ctexart}
    \usepackage{listings}
    \usepackage{xcolor}
    \usepackage{graphicx}
    \usepackage{pythonhighlight}
    \usepackage{amsmath}
    \usepackage{subfigure}
    \usepackage[left = 2cm,right = 2cm,top = 3 cm,bottom = 3cm]{geometry}
\lstset{
    numbers = left,
    frame = none,
    backgroundcolor = \color[RGB]{240,244,245},
    keywordstyle = \color[RGB]{0,0,255},
    numberstyle = \footnotesize\color{darkgray},
    commentstyle = \it\color[RGB]{255,96,96},
    stringstyle = \rmfamily\slshape\color[RGB]{255,0,255},
    showstringspaces = false,
    language=C++,
}

\begin{document}
    \title{Study Report}
    \author{Shuo Xu}
    \maketitle


    \begin{center}
        \section*{Leetcode}
    \end{center}
    \subsection*{Move Zeroes}
    \begin{itemize}
        \item Describtion
        \begin{flushleft}
            Given an array nums, write a function to move all 0's to the end of it while maintaining the relative order of the non-zero elements.

            Example:

            Input: [0,1,0,3,12]
            Output: [1,3,12,0,0]

            Note:

            1.You must do this in-place without making a copy of the array.
            
            2.Minimize the total number of operations.

        \end{flushleft}

        \item Solution
        \begin{flushleft}
            The goal of the function is to move all the zeros to the end of the given vector and keep the other elements in their original order. To do this in-place without making a copy of the array and minimize the total number of operations. I used the following solution. 

            Use a variable to record the number of zeros. Then traverses the array, and if it hits zero, the value of the variable is added one. If the value encountered is not zero, the element is moved forward according to the number of zeros currently recorded.

            After the array is traversed, all the non zero elements move forward in the original order. But there's a good chance that the last few elements aren't zero, and we need to do one more step to make the last few elements zero.
        \end{flushleft}

        \item Code
        \begin{lstlisting}
class Solution {
public:
    void moveZeroes(vector<int>& nums) {
        int zero_num = 0, len = nums.size();
        for(int i = 0; i < len ; i++){
            if(nums[i] == 0)
                zero_num ++;
            else{
                nums[i-zero_num] = nums[i];
            }
        }
        for(int i = len - zero_num; i < len; i++)
            nums[i] = 0;
        return ;
    }
};
        \end{lstlisting}
    \end{itemize}

    \begin{center}
        \section*{Others}
    \end{center}
    \begin{flushleft}
        I read a lot of related materials online this week. Now I have some ideas. Because CNN is a black box, I think if you use CNN to do the application direction, there will always be unexpected accidents, resulting in serious consequences. I searched the Internet for relevant information, but I didn't see deep learning how to deal with unexpected emergencies. It may be my improper search method, but it also reflects some problems of less knowledge. The methods I've come across are speculation, and in this case, whatever the input is, it's predicted the same way, and I think it's a tough point to deal with unexpected situations. 

        
        I am currently in the direction of medical image processing, but the relevant knowledge is still being searched.
    \end{flushleft}
\end{document}
